\section{Les hypercubes}

Un hypercube est une grille de dimension $d$ ne possédant que deux sommets selon chaque coordonnée. Ainsi, il possède $2^d$ nœuds de degré $d$. On construit un hypercube de dimension $d$ récursivement à partir de deux hypercubes de dimension $d-1$ en connectant les sommets similaires, un hypercube de dimension 0 correspondant à un nœud de calcul unique. On obtient ainsi ces quatre premières grilles :

\begin{center}

\begin{tabular}{ccccccccc}

$d = 0$ & & $d = 1$ & & $d = 2$ & & $d = 3$ & & $d = 4$ \\ 

\begin{minipage}[c]{0.05\linewidth}
\begin{center}
\begin{tikzpicture}
\SetGraphUnit{1}
\GraphInit[vstyle=Normal]
\SetUpVertex[FillColor=blue!20]
\Vertex{}
\end{tikzpicture}
\end{center}
\end{minipage}

& $\longrightarrow$ & 

\begin{minipage}[c]{0.05\linewidth}
\begin{center}
\begin{tikzpicture}
\SetGraphUnit{1}
\GraphInit[vstyle=Normal]
\SetUpVertex[FillColor=blue!20]
\Vertex{0}
\SetUpVertex[FillColor=red!20]
\SO(0){1}
\Edges(0,1)
\end{tikzpicture}
\end{center}
\end{minipage} 

& $\longrightarrow$ & 

\begin{minipage}[c]{0.1\linewidth}
\begin{center}
\begin{tikzpicture}
\SetGraphUnit{1.2}
\GraphInit[vstyle=Normal]
\SetUpVertex[FillColor=blue!20]
\Vertex{00}
\SO(00){01}
\SetUpVertex[FillColor=red!20]
\EA(00){10}
\EA(01){11}
\Edges(00,10,11,01,00)
\end{tikzpicture}
\end{center}
\end{minipage}
 
& $\longrightarrow$ &

\begin{minipage}[c]{0.15\linewidth}
\begin{center}
\resizebox{3cm}{3cm}{
\begin{tikzpicture}
\SetGraphUnit{2}
\GraphInit[vstyle=Normal]
\SetUpVertex[FillColor=blue!20]
\Vertex{000}
\EA(000){010}
\SO(000){001}
\EA(001){011}
\SetUpVertex[FillColor=red!20]
\Vertex[x=1 , y=1]{100}
\EA(100){110}
\SO(100){101}
\EA(101){111}
\Edges(000,010,011,001,000)
\Edges(000,100,110,010)
\Edges(001,101,111,011)
\Edges(100,101)
\Edges(110,111)
\end{tikzpicture}}
\end{center}
\end{minipage}

& $\longrightarrow$ &

\begin{minipage}[c]{0.3\linewidth}
\begin{center}
\resizebox{6cm}{6cm}{
\begin{tikzpicture}
\SetGraphUnit{2}
\GraphInit[vstyle=Normal]
\SetUpVertex[FillColor=blue!20]
\Vertex{0000}
\EA(0000){0010}
\SO(0000){0001}
\EA(0001){0011}
\Vertex[x=1 , y=1]{0100}
\EA(0100){0110}
\SO(0100){0101}
\EA(0101){0111}
\Edges(0000,0010,0011,0001,0000)
\Edges(0000,0100,0110,0010)
\Edges(0001,0101,0111,0011)
\Edges(0100,0101)
\Edges(0110,0111)

\SetUpVertex[FillColor=red!20]
\Vertex[x=-3.5 , y=2]{1000}
\SetGraphUnit{9}
\EA(1000){1010}
\SetGraphUnit{6}
\SO(1000){1001}
\SetGraphUnit{9}
\EA(1001){1011}
\SetGraphUnit{6}
\Vertex[x=-2 , y=3.5]{1100}
\SetGraphUnit{9}
\EA(1100){1110}
\SetGraphUnit{6}
\SO(1100){1101}
\SetGraphUnit{9}
\EA(1101){1111}
\Edges(1000,1010,1011,1001,1000)
\Edges(1000,1100,1110,1010)
\Edges(1001,1101,1111,1011)
\Edges(1100,1101)
\Edges(1110,1111)

\Edges(1000,0000)
\Edges(1100,0100)
\Edges(1001,0001)
\Edges(1101,0101)
\Edges(1011,0011)
\Edges(1111,0111)
\Edges(1010,0010)
\Edges(1110,0110)

\end{tikzpicture}}
\end{center}
\end{minipage}
\end{tabular}
\end{center}

On étiquette généralement les nœuds par une séquence de $d$ bits. Pour construire un hypercube de dimension $d$, on part d'un hypercube de dimension $d-1$ dont les sommets auront été préfixés d'un 0 (en bleu sur la figure) puis on ajoute un hypercube de dimension $d-1$ dont les sommets auront été préfixés d'un 1 (en rouge). On a alors deux propriétés intéressantes :

\begin{itemize}
\item Chaque bit correspond à une dimension de l'hypercube. Ainsi, dans notre exemple $d=3$ et par construction, le premier bit (en partant de la droite) correspond à la coordonnée classique $z$ (apparue dans l'hypercube $d=1$), le deuxième correspond à la coordonnée $y$ (issue de l'hypercube $d=2$) et enfin le premier à $x$. Les étiquettes ont donc une forme « $xyz$ ».
\item Un nœud est voisin d'un autre nœud si et seulement si leurs étiquettes diffèrent d'un seul bit.

\end{itemize}