\chapter{Les hypercubes}
\label{sec:unchapitre}

Lorem ipsum dolor sit amet, consectetur adipiscing elit. Sed non risus. Suspendisse lectus tortor, dignissim sit amet, adipiscing nec, ultricies sed, dolor. Cras elementum ultrices diam. Maecenas ligula massa, varius a, semper congue, euismod non, mi. Proin porttitor, orci nec nonummy molestie, enim est eleifend mi, non fermentum diam nisl sit amet erat. Duis semper. Duis arcu massa, scelerisque vitae,  convallis sollicitudin purus. Praesent aliquam, enim at fermentum mollis, ligula massa adipiscing nisl, ac euismod nibh nisl eu lectus. Fusce vulputate sem at sapien. Vivamus leo. Aliquam euismod libero eu enim. Nulla nec felis sed leo placerat imperdiet. Aenean suscipit nulla in justo. Suspendisse cursus rutrum augue. Nulla tincidunt tincidunt mi. Curabitur iaculis, lorem vel rhoncus faucibus, felis magna fermentum augue, et ultricies lacus lorem varius purus. Curabitur eu amet. Encore une citation \cite{Cadambe2008}.

\begin{figure}[htp!]
  \centering
  \setlength\figureheight{7cm}
  \setlength\figurewidth{9cm}
  % This file was created by matlab2tikz v0.2.2.
% Copyright (c) 2008--2012, Nico Schlömer <nico.schloemer@gmail.com>
% All rights reserved.
% 
% 
% 

% defining custom colors
\definecolor{mycolor1}{rgb}{0,0.75,0.75}

\begin{tikzpicture}

\begin{axis}[%
view={0}{90},
width=\figurewidth,
height=\figureheight,
scale only axis,
xmin=2, xmax=4.5,
xlabel={$\eta$},
xmajorgrids,
ymin=0.5, ymax=1,
ylabel={$d_{\text{min}}^2$},
ymajorgrids,
legend cell align=left,
legend style={align=left}]
\addplot [
color=black,
dashed,
mark=asterisk,
mark options={solid}
]
coordinates{
 (2,1)(2.1,1)(2.2,1)(2.3,1)(2.4,1)(2.5,1)(2.6,0.937749781479547)(2.7,0.890900393128398)(2.8,0.864988513955105)(2.9,0.827013168393703)(3,0.811347612650328)(3.1,0.792559278041243)(3.2,0.765840563467819)(3.3,0.749680961469385)(3.4,0.741947149227874)(3.5,0.740609493518419)(3.6,0.732128087463441)(3.7,0.717775843626632)(3.8,0.699687461812158)(3.9,0.685018622769455)(4,0.673439611642851)(4.1,0.664624248264608)(4.2,0.658255928882634)(4.3,0.641702335270489)(4.4,0.608326504614558)(4.5,0.580489221369454) 
};
\addlegendentry{$\alpha\text{ =  0\%}$};

\addplot [
color=black,
dashed,
mark=x,
mark options={solid}
]
coordinates{
 (2,1)(2.1,1)(2.2,1)(2.3,1)(2.4,0.958561324724996)(2.5,0.900812804739278)(2.6,0.859608621629443)(2.7,0.828484932127753)(2.8,0.812298837741994)(2.9,0.778916291864501)(3,0.758500630955482)(3.1,0.748375165853317)(3.2,0.745960208532468)(3.3,0.738441167434538)(3.4,0.715506361296671)(3.5,0.696927131434508)(3.6,0.682276848692725)(3.7,0.671128156410174)(3.8,0.663062783265717)(3.9,0.657680299791254)(4,0.621142740976429)(4.1,0.589786339121755)(4.2,0.564530571776849)(4.3,0.54483432747474)(4.4,0.53008799514765)(4.5,0.519641830384595) 
};
\addlegendentry{$\alpha\text{ = 10\%}$};

\addplot [
color=black,
dashed,
mark=triangle,
mark options={solid}
]
coordinates{
 (2,1)(2.1,1)(2.2,1)(2.3,0.966145915091813)(2.4,0.907589260275562)(2.5,0.862273165052718)(2.6,0.833762738286283)(2.7,0.797262289343802)(2.8,0.774689700869446)(2.9,0.763077871790574)(3,0.759584455148894)(3.1,0.735410358863577)(3.2,0.713220246811223)(3.3,0.695713299974315)(3.4,0.682371019886023)(3.5,0.672682085917092)(3.6,0.6661550402729)(3.7,0.644666127799479)(3.8,0.610083129739041)(3.9,0.582172698611821)(4,0.560333265725228)(4.1,0.543883933286703)(4.2,0.532098369213191)(4.3,0.524242326405)(4.4,0.519608701974017)(4.5,0.517545187250875) 
};
\addlegendentry{$\alpha\text{ = 20\%}$};

\addplot [
color=black,
dashed,
mark=triangle,
mark options={solid,,rotate=180}
]
coordinates{
 (2,1)(2.1,1)(2.2,0.995488894312993)(2.3,0.930050749246739)(2.4,0.882604857341179)(2.5,0.840148695151764)(2.6,0.807621264874927)(2.7,0.787889977099099)(2.8,0.777972678915356)(2.9,0.750463202108443)(3,0.726620292578349)(3.1,0.707917379352703)(3.2,0.693763185722015)(3.3,0.683575144048861)(3.4,0.676795290182409)(3.5,0.663350261880571)(3.6,0.627666127013326)(3.7,0.598755039468926)(3.8,0.575986310488554)(3.9,0.558651995817327)(4,0.546003746104731)(4.1,0.537291509323841)(4.2,0.531798375059385)(4.3,0.528867181690889)(4.4,0.527917002741411)(4.5,0.528450017604181) 
};
\addlegendentry{$\alpha\text{ = 30\%}$};

\addplot [
color=black,
dashed,
mark=o,
mark options={solid}
]
coordinates{
 (2,1)(2.1,1)(2.2,1)(2.3,1)(2.4,1)(2.5,0.995096871086856)(2.6,0.937749790013923)(2.7,0.890900391028178)(2.8,0.864988509535523)(2.9,0.827013167946275)(3,0.811347609462027)(3.1,0.79255927917077)(3.2,0.765840564829299)(3.3,0.749680963181722)(3.4,0.741947149533667)(3.5,0.740609492450166)(3.6,0.732128080624777)(3.7,0.71777584554089)(3.8,0.699687463368726)(3.9,0.681193180471954)(4,0.640212533267028)(4.1,0.617585040920557)(4.2,0.608519007405809)(4.3,0.608298095410932)(4.4,0.608326494076335)(4.5,0.580489212682311) 
};
\addlegendentry{Mazo};

\end{axis}
\end{tikzpicture}%
  \caption{Exemple de courbe TikZ.}
  \label{fig:courbe-tikz}
\end{figure}

\section{Implémentation du tri par fusion}
Lorem ipsum dolor sit amet, consectetur adipiscing elit. Sed non risus. 

\subsection{Quelques détails sur cette méthode}
Lorem ipsum dolor sit amet, consectetuer adipiscing elit. Morbi 

\subsection{On n'est jamais très fort pour ce calcul}
Lorem ipsum dolor sit amet, consectetuer adipiscing elit. 

\begin{align}
H_{m,n,p,q} &= \DPR{\rproto_{p,q}}{\OP{H} \tproto_{m,n}}\\
&= \iint\limits_{\SET{R}^2} S_{\OP{H}}(f,\tau) \DPR{\rproto_{p,q}}{\OP{U}_{f,\tau} \tproto_{m,n}} \ud f \ud \tau.
\end{align}

\section{Implémentation d'autres algorithmes}
Lorem ipsum dolor sit amet, consectetur adipiscing elit. 
%%% Local Variables: 
%%% mode: latex
%%% TeX-master: "isae-report-template"
%%% End: 