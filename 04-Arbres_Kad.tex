\section{Les arbres et grilles d'arbres}

\subsection{Les arbres binaires}
\paragraph{Diamètre d'un arbre binaire: } Un arbre est constitué d'un ensemble de niveaux, le niveau 0 étant occupé par la racine de l'arbre.
Les niveaux se répartissent donc de 0 à p où p est par définition la hauteur ou la profondeur de l'arbre. Chaque niveau i est occupé 
par $k^{i}$  noeuds dans un arbre k-naire, par $2^{i}$ pour un arbre binaire. 
Par conséquent, à la profondeur p, un arbre binaire complet aura donc $f = 2^{p}$ 
feuilles. Un arbre binaire complet sera constitué de $2^{p+1}-1 = 2f-1$ noeuds y compris les feuilles.
La plus grande distance entre deux noeuds, ie. le diamètre de l'arbre, est donc le nombre d'arrêtes 
du chemin passant par la racine et joignant deux feuilles. Ce qui donne : \[D = 2p = 2\log_2(f)\]


puisque la profondeur est liée au nombre de feuilles par \[p = \log_2(f)\]

% $a_{n}^{m}$

\paragraph{Bissection d'un arbre binaire : } La bissection est le nombre minimal d'arrêtes à supprimer pour qu'un graphe initialement
connexe se sépare en deux composantes connexes possédant le même nombre de noeuds à une unité près.
Pour un arbre binaire, ce nombre minimal est obtenu en supprimant une des deux arrêtes liée à la racine. On trouve
donc dans ce cas une bissection égale à 1.

\begin{figure}[htp]
  \centering
  \includegraphics[width=10cm]{images/gda}
  \caption{Grille d'arbre 2-D construite sur une grille $N \times N$ avec N = 4 (a) à laquelle on connecte 4 arbres binaires horizontaux (b) et 
  4 arbres binaires verticaux (c) pour obtenir la grille d'arbres (d).}
  \label{fig:gda}
\end{figure}
\subsection{Les grilles d'arbres}

\paragraph{Définition d'une grille d'arbre bidimensionnelle :}
Une grille d'arbre bidimensionnelle, Mesh Of Trees en anglais ou MOT en abrégé, est construite à partir d'une grille
bidimensionnelle reliée par des arbres binaires verticalement et horizontalement comme illustré figure \ref{fig:gda}.
Les noeuds de la grille constituent les feuilles des arbres. On a donc $2N$ arbres de profondeur $p = \log_2(N) = 2$.
Chacun des $2N$ arbres est composé de $(N-1)$ noeuds internes et de $N$ feuilles. On a donc un nombre total de noeuds dans la grille
d'arbres égal à la somme des noeuds de la grille $N \times N$, soit $N^2$ et du nombre total des noeuds ajoutés par les arbres, soit $2N(N-1)$. 
Une grille d'arbres $N\times N$ est donc composée de $3N^2 - 2N$ noeuds.



\paragraph{Diamètre d'un arbre binaire: } Le diamètre est la distance ``diagonale'' entre le noeud supérieur gauche et inférieur droit. La géométrie
carrée implique que cette distance est le double de la distance qui sépare le noeud supérieur gauche et le noeud inférieur gauche. Or ces 2
noeuds sont les feuilles d'un arbre binaire qui compte N feuilles donc dont le diamètre est $D = 2p = 2\log_2(N)$. \textbf{Le diamètre d'une grille d'arbre}
est donc : \[D' = 4\log_2(N)\] 

\paragraph{Bissection d'une grille d'arbres : } La figure \ref{fig:gda} (c) montre qu'en déconnectant les N = 4 racines des arbres verticaux, 
on aboutit à deux sous-réseaux de même taille si on affecte une fois sur deux les noeuds racines ainsi déconnectés au
sous-réseau supérieur puis au sous réseau inférieur. \textbf{La bissection est donc N}.

\paragraph{Bilan : } Les grilles d'arbres sont des architectures performantes car elles jouissent à la fois d'un diamètre faible, donc de connexions 
plus rapides et directes, et d'une bissection élevée, donc d'une connectivité élevée entre deux parties de la structure. 
Ce qui implique un effet ``goulot d'étranglement'' moins important que dans d'autres architectures.
Elles sont en particulier plus efficace que les grilles ou les arbres simples.




\subsection{Implémentation du produit matrice-vecteur sur une grille d'arbres : }

On suppose qu'on dispose d'une grille d'arbres $N \times N$ sur laquelle on cherche à effectuer le produit matrice-vecteur Y = AX où $A = (a_{ij})$, $X = (x_i)$
et $Y=y_j$ avec $0\leq i,j\leq N-1$. 


\begin{enumerate}
 \item On introduit $x_i$ par la racine de l'arbre horizontal $i$ non représenté sur la figure \ref{fig:gda2}(b), $0\leq i\leq N-1$. 
 \item Les $x_i$ sont transmis aux feuilles au travers des arbres horizontaux de telle 
 sorte que chaque feuille de l'arbre de la ligne $i$ 
 reçoit $x_i$ à l'étape $\log N$.
 \item On introduit $(a_{ij})$ dans la feuille $(i,j)$ via les entrées $I_i$ à l'étape $\log N$. 
 \item La feuille $(i,j)$ peut alors calculer le produit $a_{ij}x_j$.
 \item Les résultats sont sommés vers la racine des arbres verticaux.
\end{enumerate}



\begin{figure}[htp]
  \centering
  \includegraphics[width=15cm]{images/gda2}
  \caption{Grille d'arbres pour illustrer le principe du calcul d'un produit matrice-vecteur.}
  \label{fig:gda2}
\end{figure}


Après $2\log N$ étapes, la valeur \[y_i = \sum_{j=1}^{N} a_{ij}x_j\] est disponible dans 
la racine de la $i^{eme}$ colonne désignée par $O_i$ sur la figure \ref{fig:gda2} (b). 
