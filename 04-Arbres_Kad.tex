\chapter{Les arbres et grilles d'arbres}
\label{sec:unchapitre}

Lorem ipsum dolor sit amet, consectetur adipiscing elit. Sed non risus. Suspendisse lectus tortor, dignissim sit amet, adipiscing nec, ultricies sed, dolor. Cras elementum ultrices diam. Maecenas ligula massa, varius a, semper congue, euismod non, mi. Proin porttitor, orci nec nonummy molestie, enim est eleifend mi, non fermentum diam nisl sit amet erat. Duis semper. Duis arcu massa, scelerisque vitae,  convallis sollicitudin purus. Praesent aliquam, enim at fermentum mollis, ligula massa adipiscing nisl, ac euismod nibh nisl eu lectus. Fusce vulputate sem at sapien. Vivamus leo. Aliquam euismod libero eu enim. Nulla nec felis sed leo placerat imperdiet. Aenean suscipit nulla in justo. Suspendisse cursus rutrum augue. Nulla tincidunt tincidunt mi. Curabitur iaculis, lorem vel rhoncus faucibus, felis magna fermentum augue, et ultricies lacus lorem varius purus. Curabitur eu amet. Encore une citation \cite{Cadambe2008}.

\begin{figure}[htp!]
  \centering
  \setlength\figureheight{7cm}
  \setlength\figurewidth{9cm}
  \input{images/tikz_plot}
  \caption{Exemple de courbe TikZ.}
  \label{fig:courbe-tikz}
\end{figure}

\section{Diamètre et bissection}
d'une grille d'arbres bidimentionnelles


\subsection{}
Lorem ipsum dolor sit amet, consectetuer adipiscing elit.

\subsection{On n'est jamais très fort pour ce calcul}
Lorem ipsum dolor sit amet, consectetuer adipiscing elit. 

\begin{align}
H_{m,n,p,q} &= \DPR{\rproto_{p,q}}{\OP{H} \tproto_{m,n}}\\
&= \iint\limits_{\SET{R}^2} S_{\OP{H}}(f,\tau) \DPR{\rproto_{p,q}}{\OP{U}_{f,\tau} \tproto_{m,n}} \ud f \ud \tau.
\end{align}

\section{Implémentation d'un produit matrice-vecteur}
Lorem ipsum dolor sit amet, consectetur adipiscing elit. Sed non risus. 

%%% Local Variables: 
%%% mode: latex
%%% TeX-master: "isae-report-template"
%%% End: 