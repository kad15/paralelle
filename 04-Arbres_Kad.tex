\section{Les arbres et grilles d'arbres}

\paragraph{Diamètre : } Un arbre est constitué d'un ensemble de niveaux, le niveau 0 étant occupé par la racine de l'arbre.
Les niveaux se répartissent donc de 0 à h où h est par définition la hauteur de l'arbre. Chaque niveau i est occupé 
par $k^{i}$  noeuds dans un arbre k-naire. Un arbre binaire complet aura donc n = $2^{h}$ feuilles qui représente aussi le nombre de processeurs. 
La plus grande distance entre deux noeuds, ie. le diamètre de l'arbre, est donc le nombre d'arrêtes du chemin passant par la racine et joignant deux feuilles donc D = 2h = 2 * log2 n. 

% $a_{n}^{m}$

\paragraph{Bissection : } La bissection est le nombre minimal d'arrêtes à supprimer pour qu'un graphe initialement
connexe se sépare en deux composantes connexes posédant le même nombre de noeuds à une unité près.
Pour un arbre binaire, ce nombre minimal est obtenu en supprimant une des deux arrêtes liées à la racine. On trouve
donc dans ce cas une bissection égale à 1.