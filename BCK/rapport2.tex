\documentclass[a4paper,10pt]{article}
\usepackage[utf8x]{inputenc}
\usepackage[T1]{fontenc}
\usepackage[french]{babel} 
\usepackage{lmodern} % Pour changer le pack de police
\renewcommand*\familydefault{\sfdefault}
\usepackage{makeidx}
\usepackage{amsthm}
\usepackage{amsmath}
\usepackage{amssymb}
\usepackage{mathrsfs}
\usepackage{geometry}
%\usepackage{graphicx}
\usepackage{graphbox}
\usepackage{supertabular}
\usepackage{tabularx}
\usepackage{longtable}
\usepackage{pdflscape}
\geometry{hmargin=2cm,vmargin=1.5cm}

\usepackage{booktabs}
\usepackage{tabularx}
\usepackage[table]{xcolor}
\usepackage{ltablex}
\usepackage{float}
\usepackage{url}

% \usepackage{minted}

\usepackage[titletoc,toc,title,page]{appendix}
\renewcommand{\appendixtocname}{Annexes}
\renewcommand{\appendixpagename}{Annexes}

\usepackage{standalone}
\usepackage{ifthen}
\usepackage{xstring}
\usepackage{calc}
\usepackage{pgfopts}
\usepackage{tikz}
\usetikzlibrary{positioning,shapes,shadows,arrows}

\usepackage{algorithm}
\usepackage{algorithmic}

\title{Calcul parallèle\\Topologie des architectures parallèles}
\author{Florian Barbarin\\
	Abdelkader Beldjilali\\
   Nicolas Holvoet
    }
\date{\today}
% \makeindex

\theoremstyle{definition}
\newtheorem*{dfn}{Définition}

\theoremstyle{remark}
\newtheorem*{rem}{Remarque}

\theoremstyle{remark}
\newtheorem*{ex}{Exemple}

\usepackage{tkz-graph}

\begin{document}

\maketitle
\vfill
\begin{center}
\includegraphics[width=0.5\textwidth]{images/enac.png}
\end{center}


\pagebreak

\tableofcontents

\pagebreak
\section{Algorithmes de Dijkstra}
Edgser Wybe Dijkstra (EWD), Physicien Néerlandais reconverti à l'informatique en 1955, a proposé en 1959 un algorithme de recherche de chemin minimum 
dans un graphe  dont  la complexité est en O(n). 

\begin{figure}[htp]
  \centering
  \includegraphics[width=4cm]{images/Edsger_Wybe_Dijkstra}
  \caption{Edgser Wybe Dijkstra (1930-2002)}
  \label{fig:une-autre-image}
\end{figure}

On doit à Dijkstra, qui avait la réputation d'avoir mauvais caractère et de présenter une allergie au ``GOTO'',
quelques citations\footnote{source : \url{https://fr.wikipedia.org/wiki/Edsger_Dijkstra}} telles que :


\begin{quote}
\textit{« Il est pratiquement impossible d'enseigner la bonne programmation aux étudiants 
qui ont eu une exposition antérieure au BASIC : comme programmeurs potentiels, 
ils sont mentalement mutilés, au-delà de tout espoir de régénération. »}
\end{quote}

\begin{quote}
\textit{« Le plus court chemin d'un graphe n'est jamais celui que l'on croit, 
il peut surgir de nulle part, et la plupart du temps, il n'existe pas. »}
\end{quote}

\begin{quote}
\textit{« La programmation par objets est une idée exceptionnellement mauvaise qui ne pouvait naître qu'en Californie. »}
\end{quote}


L'algorithme donne le plus court chemin de la source à \textit{tous les sommets} d'un graphe 
connexe pondéré (orienté ou non) dont le poids lié aux arêtes est positif ou nul.



%  \cite{Roque2012,Roque2012b,Roque2012c,Roque2012d}. 
 


\begin{figure}[htp]
  \centering
  \includegraphics[width=15cm]{images/algo_dij}
  \caption{Exemple de calcul des plus courts chemins à partir du noeud A.}
  \label{fig:une-autre-image}
\end{figure}

L'algorithme de Disjkstra est un algorithme glouton qui utilise l'hypothèse qu'une décision prise sur la base
d'un critère d'optimalité locale conduira à un optimum global. Ainsi, à chaque itération, l'algorithme choisit le noeud
du réseau dont la distance au noeud de départ est la plus faible.
% \begin{figure}[htp]
%   \centering
%   \input{images/tikz_diagram}
%   \caption{Exemple de diagramme TikZ.}
%   \label{fig:une-image}
% \end{figure}

\section{Algorithme A*}
Lorem ipsum dolor sit amet, consectetur adipiscing elit. Sed non risus. Suspendisse l
\section{Utilisation}

Dijkstra est utilisé dans le routaghe dynamique OSPF
Comment les utiliser pour transférer de façon optimale une donnée d'un noeud à un autre. 

% \begin{table}[ht]
%   \begin{center}
%     \begin{tabular}{|c|c|c|c|c|}
%       \hline
%       & $h(t,\tau)$ & $S_{\OP{H}}^{(\alpha)} (f,\tau)$ & $L_{\OP{H}}^{(\alpha)} (\nu,t)$ & $H^{(\alpha)}(f,\nu)$ \\
%       \hline
%       LTI & $q(\tau)$ & $q(\tau) \delta(f)$ & $Q(\nu)$ & $Q(\nu) \delta(\nu-f)$ \\
%       \hline
%       LFI & $m(t) \delta(\tau)$ & $M(f) \delta(\tau)$ & $m(t)$ & $M(f)$\\
%       \hline
%       identité & $\delta(t)$ & $\delta(f)\delta(\tau)$ & $1$ & $\delta(\nu-f)$\\
%       \hline
%     \end{tabular}
%     \caption{Exemple de tableau.}
%     \label{tab:un-tableau}
%   \end{center}
% \end{table}

Lorem ipsum dolor sit amet, consectetur adipiscing elit. Sed non risus. Suspendisse lectus tortor, dignissim sit amet, adipiscing nec, ultricies sed, 
\begin{figure}[htp]
  \centering
  \includegraphics[width=4cm]{images/bitmap_image}
  \caption{Exemple d'image au format JPG.}
  \label{fig:une-autre-image}
\end{figure}


%%% Local Variables: 
%%% mode: latex
%%% TeX-master: "isae-report-template"
%%% End: 
% \section{Algorithme de Dijkstra}
% Edgser Wybe Dijkstra (EWD), Physicien Néerlandais reconverti à l'informatique en 1955, a proposé en 1959 un algorithme de recherche de chemin minimum 
% dans un graphe  dont  la complexité est en O(n). 
\section{Les grilles}

\begin{dfn}[Grille]
Une grille de dimension $d$ possédant $N$ nœuds suivant chaque coordonnée est le produit cartésien de $d$ chaines ($d>1$) de $N$ sommets. On note cette grille $M(N)^d$ que l'on dira \textit{de coté N}.
\end{dfn}




\begin{rem}

Si l'on considère le produit cartésien de deux graphes, le graphe résultant est tel que :
\begin{itemize}
\item l'ensemble de ses nœuds est le produit cartésien des nœuds des deux premiers graphes;
\item deux de ses nœuds sont voisins s'ils sont composés de nœuds qui étaient voisins dans l'un des deux premiers graphes.
\end{itemize}

\end{rem}

\begin{ex}

Soient deux chaines composées des nœuds appartenant à l'ensemble $E = \{A, B, C, D\}$. Le produit cartésien de ces deux chaines ($d=2$) de taille $N = Card(E) = 4$ :

\begin{center}

\begin{minipage}[c]{.5\linewidth}
\begin{minipage}[c]{.2\linewidth}

\begin{tikzpicture}
\SetGraphUnit{1}
\GraphInit[vstyle=Normal]
\Vertex{A}
\EA(A){B} \EA(B){C} \EA(C){D}
\Edges(A,B,C,D)
\end{tikzpicture}
\end{minipage}
\hfill
\begin{minipage}[c]{.2\linewidth}
\begin{tikzpicture}
\SetGraphUnit{1}
\GraphInit[vstyle=Normal]
\Vertex{A}
\SO(A){B} \SO(B){C} \SO(C){D}
\Edges(A,B,C, D)
\end{tikzpicture}

\end{minipage}
\end{minipage}
\end{center}

A pour résultat la grille $M(4)^2$  : 

\begin{center}

\begin{tikzpicture}
\SetGraphUnit{1.2}
\GraphInit[vstyle=Normal]
\Vertex{AA}
\EA(AA){BA} \EA(BA){CA} \EA(CA){DA}
\SO(AA){AB}
\EA(AB){BB} \EA(BB){CB} \EA(CB){DB}
\SO(AB){AC}
\EA(AC){BC} \EA(BC){CC} \EA(CC){DC}
\SO(AC){AD}
\EA(AD){BD} \EA(BD){CD} \EA(CD){DD}

\Edges(AA,BA,CA,DA)
\Edges(AB,BB,CB,DB)
\Edges(AC,BC,CC,DC)
\Edges(AD,BD,CD,DD)

\Edges(AA,AB,AC,AD)
\Edges(BA,BB,BC,BD)
\Edges(CA,CB,CC,CD)
\Edges(DA,DB,DC,DD)
\end{tikzpicture}

\end{center}

\end{ex}

\paragraph{Nombre total de nœuds :}

Il résulte de la définition ci-dessus que le nombre total de nœuds est égal au cardinal du produit cartésien de l'ensemble des nœuds de départ :

$$Card(S)= Card(\underbrace{E \times E \times ... \times E}_{\text{p fois}}) = \prod_1^p Card(E) = N^p$$

$$\text{où } S \text{ est l'ensemble des noeuds de la grille}$$

\paragraph{Nombre total d'arêtes :}

Soit $A$ l'ensemble des arêtes de la grille. Voici pour $N=2$ et $N=4$ le passage de la dimension $1$ aux dimensions supérieures ($2$ et $3$).


\begin{center}

\begin{tabular}{cccccc}

&$d = 1$ & & $d = 2$ & & $d = 3$\\ 

$N=2$& 

\begin{minipage}[c]{0.2\linewidth}
\begin{center}
\begin{tikzpicture}
\SetGraphUnit{1}
\GraphInit[vstyle=Normal]
\Vertex{A}
\EA(A){B}
\Edges(A,B)
\end{tikzpicture}
\end{center}
\end{minipage} 

& $\longrightarrow$ & 

\begin{minipage}[c]{0.2\linewidth}
\begin{center}
\begin{tikzpicture}
\SetGraphUnit{1.2}
\GraphInit[vstyle=Normal]
\Vertex{AA}
\EA(AA){BA}
\SO(AA){AB}
\EA(AB){BB}


\Edges(AA,BA,BB,AB,AA)

\end{tikzpicture}
\end{center}
\end{minipage}
 
& $\longrightarrow$ &

\begin{minipage}[c]{0.2\linewidth}
\begin{center}
\resizebox{3cm}{3cm}{
\begin{tikzpicture}
\SetGraphUnit{2}
\GraphInit[vstyle=Normal]
\Vertex{AAA}
\EA(AAA){BAA}
\SO(AAA){ABA}
\EA(ABA){BBA}
\Vertex[x=1 , y=1]{AAB}
\EA(AAB){BAB}
\SO(AAB){ABB}
\EA(ABB){BBB}


\Edges(AAA,BAA,BBA,ABA,AAA)
\Edges(AAA,AAB,BAB,BAA)
\Edges(ABA,ABB,BBB,BBA)
\Edges(AAB,ABB)
\Edges(BAB,BBB)

\end{tikzpicture}}
\end{center}
\end{minipage}

\\ 

 & $Card(A) = 1$ & & $Card(A) = 4$ & & $Card(A) = 12$ \\
 &  & & & & \\

$N = 3$ &

\begin{minipage}[c]{0.2\linewidth}
\begin{center}
\begin{tikzpicture}
\SetGraphUnit{1}
\GraphInit[vstyle=Normal]
\Vertex{A}
\EA(A){B} \EA(B){C}
\Edges(A,B,C)
\end{tikzpicture}
\end{center}
\end{minipage}

&
$\longrightarrow$
& 
\begin{minipage}[c]{0.2\linewidth}
\begin{center}
\begin{tikzpicture}
\SetGraphUnit{1.2}
\GraphInit[vstyle=Normal]
\Vertex{AA}
\EA(AA){BA} \EA(BA){CA}
\SO(AA){AB}
\EA(AB){BB} \EA(BB){CB}
\SO(AB){AC}
\EA(AC){BC} \EA(BC){CC}


\Edges(AA,BA,CA)
\Edges(AB,BB,CB)
\Edges(AC,BC,CC)

\Edges(AA,AB,AC)
\Edges(BA,BB,BC)
\Edges(CA,CB,CC)
\end{tikzpicture}
\end{center}
\end{minipage}

& $\longrightarrow$ &

\begin{minipage}[c]{0.2\linewidth}
\begin{center}
\resizebox{3.5cm}{3.5cm}{
\begin{tikzpicture}
\GraphInit[vstyle=Normal]
\SetGraphUnit{2.5}
%\draw[help lines] (0,-3) grid (2,2);

\Vertex[x=2 , y=-3]{ACC}
\EA(ACC){BCC} \EA(BCC){CCC}
\Edges(ACC,BCC,CCC)

\Vertex[x=1 , y=-4]{ACB}
\EA(ACB){BCB} \EA(BCB){CCB}
\Edges(ACB,BCB,CCB)

\Vertex[x=2 , y=-0.5]{ABC}
\EA(ABC){BBC} \EA(BBC){CBC}
\Edges(ABC,BBC,CBC)

\Vertex[x=1 , y=-1.5]{ABB}
\EA(ABB){BBB} \EA(BBB){CBB}
\Edges(ABB,BBB,CBB)

\Vertex[x=2 , y=2]{AAC}
\EA(AAC){BAC} \EA(BAC){CAC}
\Edges(AAC,BAC,CAC)

\Vertex[x=1 , y=1]{AAB}
\EA(AAB){BAB} \EA(BAB){CAB}
\Edges(AAB,BAB,CAB)

\Vertex{AAA}
\EA(AAA){BAA} \EA(BAA){CAA}
\Edges(AAA,BAA,CAA)

\SO(AAA){ABA} \EA(ABA){BBA} \EA(BBA){CBA}

\SO(ABA){ACA} \EA(ACA){BCA} \EA(BCA){CCA}
\Edges(ABA,ABB,ABC)
\Edges(BBA,BBB,BBC)
\Edges(CBA,CBB,CBC)
\Edges(AAA,AAB,AAC)
\Edges(BAA,BAB,BAC)
\Edges(CAA,CAB,CAC)
\Edges(ABA,BBA,CBA)
\Edges(ACA,BCA,CCA)
\Edges(AAA, ABA, ACA,ACA,ACB,ACC, ABC, AAC)
\Edges(BAA, BBA,BCA,BCB,BCC,BBC, BAC)
\Edges(CAA, CBA, CCA,CCB,CCC, CBC, CAC)
\Edges(CAB,CBB,CCB)
\Edges(BAB,BBB,BCB)
\Edges(AAB,ABB,ACB)
\end{tikzpicture}
}
\end{center}
\end{minipage}\\

 & $Card(A) = 2$ & & $Card(A) = 12$ & & $Card(A) = 54$ \\


\end{tabular}

\end{center}

On remarque à présent que, pour $N$ fixé, le passage d'une dimension $d$ à une dimension $d+1$ se fait en deux étapes :
\begin{itemize}
\item on "copie" $N$ fois la grille de dimension $d$;
\item on relie, par une arête, les nœuds de la grille $1$ avec les nœuds correspondants de la grille $2$ puis les nœuds de la grille $2$ avec les nœuds correspondants de la grille $3$, et ainsi de suite jusqu'à la grille $N$.

Cette méthode de construction nous donne une relation de récurrence pour le nombre d'arêtes : 

$$Card(A)_{d+1} = \underbrace{N \times Card(A)_d}_{\text{On copie }N\text{ fois la grille de dimension }d} + \underbrace{(N-1)\times Card(S)_d}_{\text{On relie les arêtes de chaque grille}}$$

On ne peut aisément déterminer une expression générale de la suite ci-dessus. Or, les exemples précédents nous permettent de déduire une expression du nombre d'arêtes en fonction de $d$ et de $N$ : 

$$Card(A)_d = d\times (N-1)\times N^{d-1}$$

Nous pouvons démontrer par récurrence cette expression.

\begin{proof}[Preuve de la relation]
\item
\paragraph{Initialisation :} 
Pour $d=1$, on a $Card(A)_1 = 1\times (N-1)\times N^{1-1} = N-1$ ce qui correspond bien au nombre d'arêtes dans une chaine.

\paragraph{Hypothèse de récurrence :}
On fait l'hypothèse qu'il existe un rang $d$ tel que $Card(A)_d = d\times (N-1)\times N^{d-1}$. Montrons que cette relation est vraie au rang $d+1$.

\paragraph{Hérédité :} Nous avons :
\begin{align*}
Card(A)_{d+1} & = N \times Card(A)_d + (N-1)\times N^d \\
& = N\times d\times (N-1)\times N^{d-1} + N^d\times (N-1)\\
& = N^d\times d\times (N-1) + N^d\times (N-1)\\
& = (d+1)\times (N-1)\times N^d
\end{align*}

Nous retrouvons bien l'hypothèse de récurrence au rang $d+1$. On en déduit que $\forall d \in \mathbb{N^*}$, on a $Card(A)_d = d\times (N-1)\times N^{d-1}$.
\end{proof}

D'après l'ensemble des éléments qui précèdent, le nombre d'arêtes d'une grille est telle que :

$$\forall d \in \mathbb{N^*} \text{, }Card(A)_d = d\times (N-1)\times N^{d-1} $$ 

\end{itemize}

\section{Les arbres et grilles d'arbres}

\paragraph{Diamètre : } Un arbre est constitué d'un ensemble de niveaux, le niveau 0 étant occupé par la racine de l'arbre.
Les niveaux se répartissent donc de 0 à h où h est par définition la hauteur de l'arbre. Chaque niveau i est occupé 
par $k^{i}$  noeuds dans un arbre k-naire. Un arbre binaire complet aura donc n = $2^{h}$ feuilles qui représente aussi le nombre de processeurs. 
La plus grande distance entre deux noeuds, ie. le diamètre de l'arbre, est donc le nombre d'arrêtes du chemin passant par la racine et joignant deux feuilles donc D = 2h = 2 * log2 n. 

% $a_{n}^{m}$

\paragraph{Bissection : } La bissection est le nombre minimal d'arrêtes à supprimer pour qu'un graphe initialement
connexe se sépare en deux composantes connexes posédant le même nombre de noeuds à une unité près.
Pour un arbre binaire, ce nombre minimal est obtenu en supprimant une des deux arrêtes liées à la racine. On trouve
donc dans ce cas une bissection égale à 1.
\section{Les hypercubes}

test push
\end{document}
